\documentclass[10pt]{beamer}
\usepackage[utf8]{inputenc}
\usepackage{graphicx}
\usepackage{tikz-cd}
\usepackage{amsmath}
\usepackage{amssymb}
\usepackage{quiver}
\usepackage{babel}
\usepackage[backend=bibtex,citestyle=authoryear]{biblatex}

\addbibresource{references.bib}

\usetheme{metropolis}
\metroset{progressbar=frametitle}
\metroset{numbering=none}
\usecolortheme{beaver}

\titlegraphic{
  \includegraphics[width=3cm,height=1.6cm,keepaspectratio]{R-removebg-preview.png}\hfill
  \vfill
}

% Define the custom red color
\definecolor{customred}{HTML}{b70e39}
\definecolor{dark_customred}{HTML}{9d1c3d}

% Custom progress bar thickness globally
\makeatletter
\setlength{\metropolis@progressinheadfoot@linewidth}{1.2pt} % Adjust the thickness here
\makeatother

% Customize the theme colors
\setbeamercolor{structure}{fg=customred}
\setbeamercolor{frametitle}{fg=customred}
\setbeamercolor{title}{fg=customred}
\setbeamercolor{section in sidebar}{fg=customred}
\setbeamercolor{subsection in sidebar}{fg=customred}
\setbeamercolor{title in sidebar}{fg=customred}
\setbeamercolor{author in sidebar}{fg=black}

% Define the logo
\logo{\includegraphics[height=1cm]{R-removebg-preview.png}}

% Remove the page count
\setbeamertemplate{footline}{}

\title{Where is Intelligence?}
\subtitle{Untersuchungen an Menschen oder Affen?}
\author{Daniel Schellhorn}
\institute{Uni Osnabrück}
\date{\today}

\titlegraphic{{\includegraphics[height=1cm]{R-removebg-preview.png}}}

\begin{document}

\maketitle


% Table of Contents
\begin{frame}[t]
    \frametitle{Table of Contents}
    \footnotesize
    \begin{columns}
      \begin{column}{1\textwidth}
        \tableofcontents[sections={1-3}]
      \end{column}
    \end{columns}
\end{frame}

    

\section{The Intelligence of Köhler}

\begin{frame}
    \subsection{Definition of Intelligence}
    \frametitle{Definition of Intelligence}
    \begin{itemize}
        \item Intelligence in animals involves problem-solving and the ability to find roundabout ways to achieve goals.
        \item It requires the animal to navigate obstacles and find indirect routes when direct paths are blocked.
        \item True intelligence is marked by the ability to perform smooth, continuous movements towards the objective.
    \end{itemize}
    \end{frame}
    
    \subsection{Characteristics of Intelligent Behavior}
    \begin{frame}
    \frametitle{Characteristics of Intelligent Behavior}
    \begin{itemize}
        \item Smoothness and continuity in movement when solving a problem.
        \item Ability to adapt to variations in the environment and obstacles.
        \item Use of prior knowledge and experience to navigate unseen parts of the path.
    \end{itemize}
    \end{frame}
    
    \subsection{Distinguishing Genuine Intelligence from Chance}
    \begin{frame}
    \frametitle{Distinguishing Genuine Intelligence from Chance}
    \begin{itemize}
        \item Genuine solutions are continuous and unified, both in space and time.
        \item Chance solutions consist of disjointed, independent movements.
        \item Observing the animal's behavior helps distinguish between true intelligence and chance.
    \end{itemize}
    \end{frame}
    
    \subsection{Role of Insight in Problem-Solving}
    \begin{frame}
    \frametitle{Role of Insight in Problem-Solving}
    \begin{itemize}
        \item Insight involves a sudden understanding of the solution, marked by a noticeable behavioral change.
        \item Examples include sudden changes in direction or expressions of realization (e.g., facial expressions in children).
        \item Insightful behavior contrasts with the trial-and-error approach seen in less intelligent animals.
    \end{itemize}
    \end{frame}
    
    \subsection{Thorndike's Experiments on Animal Intelligence}
    \begin{frame}
    \frametitle{Thorndike's Experiments on Animal Intelligence}
    \begin{itemize}
        \item Thorndike's experiments showed animals' difficulties in solving problems without a full view of the situation.
        \item His tests suggested that animals do not reason like humans but rely on experiential linking of impulses and perceptions.
        \item Prolonged learning was often necessary before animals developed the correct action.
    \end{itemize}
    \end{frame}
    
    \subsection{Critique of Thorndike's Methodology}
    \begin{frame}
    \frametitle{Critique of Thorndike's Methodology}
    \begin{itemize}
        \item Thorndike's experiments often did not allow animals to fully survey the problem.
        \item Essential parts of the experimental apparatus were sometimes hidden, limiting the animals' ability to use their intelligence.
        \item Observing the complete experimental setup is crucial for evaluating true intelligent behavior.
    \end{itemize}
    \end{frame}
    
    \subsection{Conclusion on Animal Intelligence}
    \begin{frame}
    \frametitle{Conclusion on Animal Intelligence}
    \begin{itemize}
        \item True intelligence in animals involves insight, adaptation, and the ability to navigate complex environments.
        \item Chance and trial-and-error play roles, but genuine intelligence is marked by smooth and continuous problem-solving behavior.
        \item Further research and better experimental designs are necessary to fully understand animal intelligence.
    \end{itemize}
    \end{frame}




\section{Reflection on his Intelligence} 

\subsection{Introduction}
\begin{frame}
    \frametitle{Introduction}
    \begin{itemize}
        \item The definition of intelligence can vary significantly depending on cultural perspectives and the criteria used to measure it.
        \item Traditional Western perspectives often focus on cognitive abilities and problem-solving.
        \item Other viewpoints, such as those from certain indigenous cultures, might place greater value on holistic understanding and interconnectedness with nature.
    \end{itemize}
\end{frame}

\subsection{Animal Intelligence and Adaptation}
\begin{frame}
    \frametitle{Animal Intelligence and Adaptation}
    \begin{itemize}
        \item Cognitive problem-solving abilities and adaptability within complex environments are key indicators of intelligence.\footcite{zuberbuhler2000cognitive}
        \item This includes the ability to plan, use tools, and learn from experience in a smooth, continuous manner rather than through trial and error.
        \item Falcons and other birds of prey demonstrate remarkable intelligence through their precise hunting techniques and navigation skills.\footcite{emery2004cognitive}
        \item Their ability to spot prey from great distances and react swiftly suggests a high level of specialized intelligence adapted to their ecological niche.
    \end{itemize}
\end{frame}

\subsection{Falcon Intelligence}
\begin{frame}
    \frametitle{Falcon Intelligence}
    \begin{itemize}
        \item Falcons exhibit remarkable navigation and hunting skills, indicating a high level of specialized intelligence.\footcite{ratcliffe2007predation}
        \item Their acute vision and precise hunting techniques are adapted to their ecological niche.
        \item This intelligence is often considered instinctual and specialized rather than broad and adaptable.
    \end{itemize}
\end{frame}

\subsection{Cultural Perspectives on Intelligence}
\begin{frame}
    \frametitle{Cultural Perspectives on Intelligence}
    \begin{itemize}
        \item Indigenous tribes, such as those in Peru, view intelligence as a holistic understanding of being part of a greater system (Gaia or Pachamama).\footcite{descola1996nature}
        \item This worldview suggests that true intelligence lies in recognizing and living in harmony with the interconnectedness of all life forms.
        \item This perspective challenges the hierarchical view of intelligence that places humans at the top.
        \item It values the wisdom and balance observed in natural systems and the non-human entities within them.\footcite{berkes1999sacred}
    \end{itemize}
\end{frame}

\subsection{Indigenous Perspectives on Intelligence}
\begin{frame}
    \frametitle{Indigenous Perspectives on Intelligence}
    \begin{itemize}
        \item Some indigenous cultures, like the Shipibo in Peru, view intelligence hierarchically with humans at the lowest level.\footcite{lovelock1972gaia}
        \item They believe true intelligence involves understanding and integrating with the natural world, a concept similar to the Gaia hypothesis.
        \item This perspective challenges conventional hierarchies of intelligence, emphasizing harmony with nature.
    \end{itemize}
\end{frame}

\subsection{Reevaluating Intelligence}
\begin{frame}
    \frametitle{Reevaluating Intelligence}
    \begin{itemize}
        \item The concept of intelligence can vary widely across cultures.\footcite{nisbett2003geography}
        \item Western definitions often emphasize cognitive problem-solving and adaptability.
        \item Other cultures may prioritize holistic understanding and integration with the environment.
    \end{itemize}
\end{frame}

\subsection{Defining Intelligence}
\begin{frame}
    \frametitle{Defining Intelligence}
    \begin{itemize}
        \item Traditional definitions of intelligence often emphasize cognitive functions such as memory, reasoning, problem-solving, and learning.\footcite{gardner1983frames}
        \item Humans excel in these areas, particularly in abstract thinking and symbolic communication.
        \item However, this definition may be too narrow to encompass all forms of intelligence observed in nature.
        \item By including ecological and holistic intelligence, as seen in animals like falcons and in indigenous knowledge systems, we gain a broader and more inclusive understanding of intelligence.\footcite{holling2001understanding}
    \end{itemize}
\end{frame}

\subsection{Conclusion}
\begin{frame}
    \frametitle{Conclusion}
    \begin{itemize}
        \item Intelligence is a multifaceted concept that can include cognitive abilities, specialized skills, and holistic understanding.
        \item Animal intelligence can be broad and adaptable or specialized and instinctual.
        \item Cultural perspectives offer valuable insights that challenge conventional definitions and hierarchies of intelligence.
    \end{itemize}
\end{frame}

\section{Discussion on our Intelligence}
\subsection{Where is Intelligence?}
\begin{frame}
\frametitle{Where is Intelligence?}
\[
	\resizebox{0.8\textwidth}{!}{
	\begin{tikzcd}[ampersand replacement=\&,sep=normal, cells={nodes={draw, minimum width=1cm, minimum height=0.5cm, anchor=center}}]
        \&\&\&\& Self \\
        \\
        \&\&\&\& Intellect \\
        \\
        \&\&\&\& {Slow\ Mind} \\
        \&\&\&\& Ego \\
        Object \&\& Body \&\&\&\& Soul \&\& Subject \\
        \&\&\&\& Socio \\
        \&\&\&\& {Fast\ Mind} \\
        \\
        \&\&\&\& Life \\
        \\
        \&\&\&\& Death
        \arrow["Motivation", from=1-5, to=7-9]
        \arrow["Decision"{description}, from=3-5, to=1-5]
        \arrow["Cognition"{description}, from=5-5, to=3-5]
        \arrow["Emotion", from=5-5, to=7-7]
        \arrow["Emotion"{description}, from=6-5, to=7-7]
        \arrow["Bindung"{description}, tail reversed, from=6-5, to=8-5]
        \arrow["Observation", from=7-1, to=1-5]
        \arrow["Sensation", from=7-1, to=7-3]
        \arrow["Preservation"', from=7-1, to=13-5]
        \arrow["Interpratation", from=7-3, to=5-5]
        \arrow["Condition"{description}, from=7-3, to=6-5]
        \arrow["Intuition"', from=7-3, to=9-5]
        \arrow["Motivation", from=7-7, to=7-9]
        \arrow["Expression"{description}, from=7-7, to=8-5]
        \arrow["{{{Sensation`}}}"{description}, from=8-5, to=7-3]
        \arrow["Emotion"', from=9-5, to=7-7]
        \arrow["Memorization"{description}, from=9-5, to=11-5]
        \arrow["Instict"{description}, from=11-5, to=13-5]
        \arrow["Motivation"', from=13-5, to=7-9]
\end{tikzcd}
}
\]
\end{frame}




\begin{frame}[allowframebreaks]
    \printbibliography
    \end{frame}
    
    
\end{document}
