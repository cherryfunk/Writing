\documentclass{article}
\usepackage[utf8]{inputenc}
\usepackage{graphicx} % For including figures

\title{Mentality of Apes}
\date{}

\begin{document}
\maketitle

\section{Roundabout Methods\protect\footnotemark} 
\footnotetext{Also called detours, roundabout ways, paths or routes, circuitous routes and indirect ways in this book. No one English word quite covers the meaning of " Uwege ” [Tr. Note.].}

When any of those higher animals, which make use of vision,
notice food (or any other objective) somewhere in their field
of vision, they tend—so long as no complications arise
—
to go after it in a straight line. We may assume that this
conduct is determined without any previous experience,
providing only that their nerves and muscles are mature
enough to carry it out.

Thus, if the principle of experimentation mentioned in
the introduction is to be applied in a very simple form, we
may use the phrases "direct way" and "roundabout way"
quite literally, and set a problem which, in place of the
biologically-determined direct way, necessitates a complicated
geometry of movement towards the objective. The direct
way is blocked in such a manner that the obstacle is quite
easily seen ; the objective remains in an otherwise free field,
but is attainable only by a roundabout route. First it is
assumed that the objective, the obstruction, and also the
total field of possible roundabout routes are in plain sight
;if the obstruction be given various forms, there will develop
also a variety of approaches to the objective, and perhaps,
at the same time, variations in the difficulties which such a
situation contains for the animal.

This test which, on nearer investigation, appears to be
the simplest and, in some respects, fundamental for theoretical problems, will, in chimpanzees from four to seven years of
age and in the form described, yield no results which cannot
be observed in their ordinary behaviour. Chimpanzees
will get round any obstruction lying between them and
their objective, if they have sufficient view of the space
in which lie the possible detours. The path may lie across
flat ground, or over trees and scaffolding, or even up under
a roof as long as they can grab hold of something. Thus in
experiments to be described later, in which the objective 
hung from the wire-roof of their playground, the first attempt
at solution often consisted in their climbing to the roof at
the first available point, and thence arriving at the hanging
cord. It required strict vigilance to eliminate from the
programme this and other detours which only climbers
like chimpanzees, and among them only the real acrobats,
like Chica, would hit upon. For it must not be assumed
that even in bodily dexterity chimpanzees are all alike.
One sees the animals twisting, bending, and turning their
bodies with equal facility according to the shape of an
entrance ; but no one expects a chimpanzee to remain helpless
before a horizontal opening in a wall, on the other side of
which his objective lies, and so it makes no impression at all on us when he makes as horizontal a shape as he can
of himself, and thus slips through. It is only when round-
about methods are tried on the lower animals, and when
you see even chimpanzees undecided, nay, perplexed to the
point of helplessness, by a seemingly minor modification
of the problem—it is only then you realize that circuitous
methods cannot in general be considered usual and matter-of-
course conduct.\footnote{Cf. the last section of this book, p. 226, seqq.
} But, as chimpanzees do not give us the
impression of any particular insight when they take a round-
about route (at any rate in the form so far discussed) no
further explanation is here required, because of the non-
theoretical form of our problem.

Meanwhile, however, in the simplest experiments of the
roundabout type, observation is so easy that a description
of such tests performed on other animals is advisable. Taking
such a simple case as an example, one becomes aware of a
factor which occurs over and over again in all difficult experi-
ments with chimpanzees, and will be more easily observed
there after it has become familiar here. Therefore the
following examples are quoted.

Near the wall of a house, a square piece of ground is fenced
off so that one side, one metre from the house, is parallel to
it, and forms with it a passage two metres long ; one end
of this passage is cut off by a railing. A mature Canary
Isle bitch is brought into this blind alley from direction A
(cf. Fig. 2), to B, where she is kept occupied with food, her
face towards the railings. When the food is nearly gone,
more is put down at the spot C, on the other side of the rail
;the bitch sees it, seems to hesitate a moment, then quickly
turns at an angle of 180 0 and is already on the run in a smooth
curve, without any interruption, out of the blind alley,
round the fence to the new food.

The same dog, on another occasion, behaved at first in
the same way. It was standing at B near a wire fence (con
structed as in Fig. 3) over which food was thrown to some
distance ; the bitch at once dashed out to it, describing a
wide bend. It is worth noting that when, on repeating this
experiment, the food was not thrown far out, but was dropped
just outside the fence, so that it lay directly in front of her,
separated only by the wire, she stood seemingly helpless,
as if the very nearness of the object and her concentration
upon it (brought about by her sense of smell) blocked the
" idea ” of the wide circle round the fence ; she pushed again
and again with her nose at the wire fence, and did not budge
from the spot.

A little girl of one year and three months, who had learned
to walk alone a few weeks before, was brought into a blind
alley, set up ad hoc (two metres long, and one and a half wide),
and, on the other side of the partition, some attractive object
was put before her eyes ; first she pushed towards the object,
i.e., against the partition, then looked round slowly, lot her
eyes run along the blind alley, suddenly laughed joyfully,
and in one movement was off on a trot round the corner
to the objective.

In similar experiments with hens, one sees that a roundabout
way is not taken as a matter of course, but is quite an achieve-
ment ; hens, in situations which are much less roundabout
than those already described, have been quite helpless
they keep rushing up against the obstruction when they see
their objective in front of them through a wire fence, rush
from one side to the other all a-fluster, and do not fare
better, even when they are familiar with the obstruction
(or the fence) and the greater part of the circuitous route,
as, for instance, round the little door of their place and
through the opening corresponding to it. Different hens
do not behave in the same way, and, if the detour is shortened
while they are still pushing against the obstacle, it can easily
be observed how first one, then another, and so on, stops
running up against the obstruction, and runs quickly round
the curve ; but some particularly ungifted specimens keep on
running up against the fence a long while even in the simplest
predicaments. The difference is very plain too, when one
notices in cases of longer roundabout routes to what an
extent chance must help to solve the problem. In their
oscillations in front oi the objective, the hens now and then
run into places from which the circuitous route is shorter
;
but this easing-up brought about by chance will have a very
different effect on different animals : one will suddenly
rush out in a closed circle, another will still zigzag helplessly
to and fro in the'" wrong ” direction. All the hens which
I observed thus managed to achieve only very “ straight ”
roundabout ways (cf. Fig. 4a in contrast to 46). Apparently
the possible detour must not begin with the direction leading
away from the objective (cf. as against this the behaviour
of the child and the dog above).

It therefore follows that for those processes which form the
basis of this small achievement, variations in the geometrical
circumstances are of the greatest importance. \footnote{In what way they are dependent can be more exactly determined,
and when this is known, definite conditions wr ill exist foi every theory
concerning the experiment.} The influence
of these circumstances will more than once be striking in the case of the anthropoids, in what are, for them, much harder
tasks.

As chance can bring the animals into more favourable
spots, it will also occasionally happen that a series of pure
coincidences will lead them from their starting-point right
up to the objective, or at least to points from which a straight
path leads to the objective. This holds in all intelligence
tests (at least in principle : for the more complex the problem
to be solved, the less likelihood is there that it will be solved
wholly by chance) ; and, therefore, we have not only to answer
the question whether an animal in an experiment will find
the roundabout way (in the wider meaning of the word) at
all, we have to add the limiting condition, that results of
chance shall be excluded. Now (if we take as examples these
experiments in roundabout ways—in the narrower sense) since
approximately the same path must be followed by the animal,
whether as the result of a succession ,of accidents, or of a
real solution of the problem, the objection will arise, that one
cannot distinguish between these two possibilities. It is
of great importance for what follows and for the psychology
of the higher animals in general, that one should not allow
oneself to be confused by such apparently " pat ” but, in
reality, false, considerations. Observation, which alone may
be admitted here, shows that there is in general a rough differ-
ence in form between genuine achievement and the imitations
of accident, and no one who has performed similar experiments
on animals (or children) will be able to disregard this difference.
The genuine achievement takes place as a single continuous
occurrence, a unity, as it were, in space as well as in time ;
in our example as one continuous run, without a second’s
stop, right up to the objective. A successful chance solution
consists of an agglomeration of separate movements, which
start, finish, start again, remain independent of one another
in direction and speed, and only in a geometrical summation
start at the starting-point, and finish at the objective. The
experiments on hens illustrate the contrast in a particularly
striking way, when the animal, under pressure of the desire
to reach the objective, first flies about uncertainly (in zigzag
movements which are shown in Fig. 4 but in not nearly
great enough confusion), and then, if one of these zigzags
leads to a favourable place, suddenly rushes along the curve
in one single unbroken run. Here, the first part of the
possible path is swallowed up in confused zigzagging, all the
rest is " genuine” —the one type of behaviour succeeding
the other so abruptly that no one could mistake the difference
in the two kinds of movements.

If the experiment has not been made often, there is the
additional fact that the moment in which a true solution
is struck is generally sharply marked in the behaviour of
the animal (or the child) by a kind of jerk : the dog
stops, then suddenly turns completely round (180 °), etc.,
the child looks about, suddenly its face lights up, and
so forth. Thus the characteristic smoothness of the true
solution is made more striking by a discontinuity at its
beginning.

I must explicitly warn my readers against the mistake
of thinking that I am implying any supernatural mode of
interpreting behaviour : any practised person can observe
this, not only in experiments on animals, but in all others.
Similar considerations have to be taken into account often
enough outside the animal world. Thus, wandering earth-
currents, and other rapidly-alternating fortuitous influences,
deflect the thread of a badly set-up electrical measuring
instrument irregularly to and fro on the scale ; but should
the thread move constantly to a certain scale division, no
physicist would mistake the evident difference, and its
meaning. In observing the Brownian movement any experi-
mental error which causes the introduction of a regular
movement into one which is normally irregular would at once
be detected, and so forth. Later on, more will be said about
this matter, the importance of which does not concern method
alone.

[Experiments in roundabout ways of the kind described
must not be confused with two other experimental methods :
\begin{enumerate}
    \item "Frogs without brain and mid-brain still get out of the
    way of obstacles" (Nagel, Physiol, des Menschen, IV, I, p. 4 ;
    A. Tschermak). Thus the animals move automatically out
    of a line of motion which would bring them into collision
    with an obstacle. Does it follow that the same frogs would
    automatically take a long way round an obstacle up to an
    objective? Obviously not. The main point in our experi-
    ment does not arise at all in the frog experiment.
    \item American
    animal psychology makes animals (or people) seek the way
    out of mazes, over the whole of which there is no general
    survey from any point inside ; the first time they get
    out is, therefore, necessarily a matter of chance, and so,
    for these scientists, the chief question is how the ex-
    perience gained in such circumstances can be applied in
    further tests.
\end{enumerate}
In intelligence tests of the nature of
our roundabout-way experiments, everything depends upon
the situation being surveyable by the subject from the
outset.]

I made the experiment more difficult for chimpanzees,
in the following way : The objective hangs in a basket from
the wire-roof and cannot be reached from the ground ; the
basket contains also several heavy stones, so that one push
of the string and basket will make the whole swing for some
little time ; the swing is so arranged that the longest sideways
movement of the basket makes it nearly reach a scaffolding.
Thus the roundabout way is easily recognizable, and available,
but only for a few moments. —(19. 1 14) —As soon as the
basket is swinging, Chica, Grande, and Tercera are let in
upon the scene. \footnote{ In the first few days these animals were far too timid to permit of the
isolation of any one of them for experimentation , this circumstance
caused the very greatest difficulties, and even after six months it was
still impossible to test Chica alone Usually in such cases I gave
Tercera or Konsul as companions ; for they were not much use anyhow
on account of shyness or laziness ; but other subjects of experiment
were sometimes similarly waste} Grande leaps for the basket from the
ground, and misses it. Chica who, in the meantime, has
quietly surveyed the situation, suddenly runs towards the
scaffolding, waits with outstretched arms for the basket, and
catches it. The experiment lasted about a minute. \footnote{ In this book I give either no times at all or the approximate time
in those instances where it bears on the subject In general the duration
of an experiment depends on so many accidental and changing circum-
stances (e g futile attempts at solution, lack of interest, depression on
account of failure or isolation, etc ) that measures of time would only
give the semblance of a quantitative method The time-data in any
of these experiments can always be judged or estimated from the
description, as far as it is important for our purposes Whether an
interval ol indifference or complaining, as often occurred, lasted three
minutes, 1 e perhaps ten times as long as the actual time of solution,
or half an hour, perhaps a thousand times the length of that, does not
matter at all In most cases the solution itself would make up any
fraction one liked of the measured "duration of the experiment".}

Repetitions with other animals (Rana, Koko) also went
so smoothly and quickly that one can probably infer that
every chimpanzee can solve this problem. Grande, who had
seen Chica’s solution, duplicated it on an immediate repetition
of the test. Judging by everything that happened later,
there is no doubt that example is not absolutely necessary,
and that, always slower than the others, Grande would,
after a little while, have seen the roundabout route of herself.

Sultan, who was not present at these experiments, was
tested with the same swing (20.1), but this time, before
he saw it, the basket was set swinging in a circle which brought
it at regular speed past a beam ; the circular swing and
the regular speed doubtless made this experiment a little
harder. Sultan looked up for a second, and followed the
basket with his eyes ; when he saw it swinging past the
beam, he was up there at once, awaiting it.

In experiments such as these, it does not matter at all
whether the point which the swing approaches remains
the same in successive experiments or not ; and neither
does it matter whether the vantage-point is a wall, a tree,
a scaffolding, or anything else. If variations of this sort
are introduced, the animal does not climb up to the spot
at which it was successful before ; it clambers with complete
certainty to the right place for the new situation. In experi-
ments as simple as this I never saw this rule broken, but in
harder tasks, mistakes involving stupid repetitions did occur.

The experiment is considerably more difficult when a
part of the problem, if possible the greater part, is not visible
from the starting-point, but is known oniy “ from experience. ”

One room of the monkey-house has a very high window,
with wooden shutters, that looks out on the playground.
The playground is reached from the room by a door, which
leads into the corridor, a short part of this corridor, and a
door opening on to the playground (cf. Fig. 5). All the
parts mentioned are well known to the chimpanzees, but
animals in that room can set only the interior. (6.3) —
Intake Sultan with me from another room of the monkey-house,
where he was playing with the others, lead him across the
corridor into that room, lean the door to behind us, go with
him to the window, open the wooden shutter a little, throw
a banana out, so that Sultan can see it disappear tlirough
the window, but, on account of its height, does not see it fall,
and then quickly close the shutter again (Sultan can only
have seen a little of the wire-roof outside). When I turn
round Sultan is already on the way, pushes the door open,
vanishes down the corridor, and is then to be heard at the
second door, and immediately after in front of the window.
I find him outside, eagerly searching underneath the window ;
the banana has happened to fall into the dark crack between
two boxes. Thus not to be able
to see the place where the ob-
jective is, and the greater part
of the possible indirect way to
it, does not seem to hinder a
solution ; if the lay of the land
be known beforehand, the in-
direct circuit through it can be
apprehended with ease.

In a very similar experiment
with the bitch already men-
tioned, she managed the same manoeuvre. From the yard
which runs straight and unencumbered around the house,
one steps through the door D into a room with its window
W looking out on to the yard Y (cf. Fig. 6) ; the bitch,
who is acquainted with the room and the yard from former
visits —she does not belong to the house —is brought through
the door D into the room, and is tempted, with food, to the
open window ; from here she can see only the tops of distant
trees, not the yard itself. The food is thrown out, and the
window at once shut. The dog jumps once against the
window-pane, then stands a moment, her head raised towards
the window, looks a second at the observer, when all at once
she wags her tail a few times, with one leap whirls round
18o°, dashes out of the door, and runs round outside, till
he is underneath the window, where she finds the food
immediately.\footnote{Somewhat different experiments in detours were made by Thorndike
(cf the work quoted below) and Hobhouse (Mind in Evolution, London
1901, p. 223 seqq.). I must add that the bitch was not brought through
the door from the window-side of the house, so that behind her she can
only have had a scent-trail as far as the door; in any case her sense of
smell was not observed to have played any part at all.
}

[Thorndike tested large numbers of dogs and cats in order
to see what there is in the wonder-stories that are told about
these domestic pets. The result was very unfavourable to
the animals, and Thorndike came to the conclusion that,
so far from “ reasoning ”, they do not even associate images
with perception, as humans do, but remain limited chiefly
to the experiential linking of mere 41
impulses ” with per-
ceptions. This investigation did what was necessary in a
negative way at the time, but, as is now being shown (also
in America), it went a little too far. The tests were based
upon those animal stories, and consequently were made so
difficult that the result was bound to fall out badly , under
the influence of the animals’ failures in these tests, Thorndike
then drew generalizations about their capacities, which
do not follow from those difficult experiments. However
stupid a dog may seem compared to a chimpanzee, we suggest
that in such simple cases as have just been described, a closer
investigation would be desirable.

Regarding their principle, I must make a further objection
to Thorndike’s experiments. They were designed as intelli-
gence tests of the same type as our own (insight or not ?)\footnote{See foot-note, p. 219.},
and ought, therefore, to have conformed to the same general
conditions, and, above all, to have been arranged so as to be
completely visible to the animals. For if essential portions
of the experimental apparatus cannot be seen by the animals,
how can they use their intelligence faculties in tackling
the situation? It is somewhat astonishing to find that
(in Thorndike’s experiments) cats and dogs were frequently
placed in cages containing the extreme end only of one or
the other mechanism, or allowing a view of ropes or other
parts of the mechanism, but from which a survey over the
whole arrangement was not possible. The task for the animal
was to let itself out of the cage by pulling or pressing the
accessible part of the mechanism ; then—the cage door
would open of itself. Thorndike also gives an account of
experiments in which the animals were let out of their cages
if they scratched or licked themselves. He contrasts these
experiments with those involving the employment of any
mechanical contrivance, as the former apparently imply
no direct connexion between cause and effect ; but the
causation is far from apparent even in the mechanistic
experiments.

In the case of the latter, there are at least various com-
ponent parts which can be treated with some amount of
insight, and it is of the highest significance to know whether
animals react differently to experimental situations which
involve a partial possibility of intelligent behaviour than
they do to such as involve none—for the difference, if any,
is obviously crucial.

The result of these experiments tends to show that pro-
longed “ learning ” is necessary before the right action
develops, in both sets —as the “ experiments with a mechan-
ism ” were far too difficult, and, in many cases, could not be
wholly surveyed either. But when once the animals have
mastered both procedures, a noticeable difference is shown
:“In all these cases ”—of the meaningless type —“ there is a
noticeable tendency ... to diminish the required action,
till it becomes a mere vestige of a lick or scratch ” —and
more especially—“ if sometimes you do not let the cat
out after this feeble reaction, it does not at once repeat
the movement, as it would do if it depressed a thumbiece, for instance, without success in getting the door
open." \footnote{Animal Intelligence, New York, 1911, p. 48}

Thorndike merely states that he cannot give a reason
for the difference of result in the two types of experiment.
As these results are among the most interesting which he
has obtained—though scarcely what we might expect from
his theory—we can only regret that he has not probed further.]

\end{document}
