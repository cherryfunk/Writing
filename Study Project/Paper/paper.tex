\documentclass[12pt]{article}
\usepackage[utf8]{inputenc}
\usepackage{amsmath}
\usepackage{amssymb}
\usepackage{graphicx}
\usepackage{geometry}
\usepackage[backend=biber, style=authortitle, citestyle=verbose-trad1]{biblatex}
\geometry{a4paper, margin=1in}
\addbibresource{references.bib}


\title{Where is Intelligence?}
\author{Daniel Schellhorn}
\date{\today}

\begin{document}

\maketitle

\section{Introduction}

Intelligence is a mystery that has been studied by psychologists, biologists, philosophers, and every other available field for years. As much as human intelligence has been researched for centuries, the study of intelligence is increasingly becoming focused on understanding intelligence in animals. This is because one now recognizes that such cognitive abilities are not solely characteristic of humans but abound in different levels through various animal species, hence questioning the linearity of human and non-human minds.

Understanding the intelligence of humans and animals is important on many levels. First, it offers insight into the evolutionary origins of cognitive abilities and problem-solving, which could indicate a biological predisposition to intelligence. Second, understanding how animals solve complex problems in adapting to their natural environments assists scientists in refining their notion of intelligence beyond the human-centered model. The research is important, most specifically, to provide detailed information on the principles underlying learning and behavior which may also extend to other general areas of study in the fields of artificial intelligence, neuroscience, and behavioral ecology.

This seminar paper looks at the multidimensionality of intelligence based on Köhler's own work with animals, particularly in his experiments with the primate species. Köhler defines intelligence as the ability of an organism to solve a problem through insight, not by trials and errors; it is on this definition that this review is based. The findings presented by Köhler are compared with other research, for instance those done by Thorndike, to point out the differences between true intelligence and chance behavior. This paper, therefore, tries to go back to the general understanding of what constitutes intelligent behavior in animals and, by extension, humans through the critical analyses of various methodologies applied in the assessment of intelligence.

\newpage

\section{The Intelligence of Köhler}

Of particular consequence in this regard are the experiments with chimpanzees, in which Köhler has endeavored to work out a rather more characteristic approach to the understanding of intelligence. According to him, the true intelligence manifests itself not so much in the fact that an animal solves its problems as in the manner of solving these problems, that is, not through random or accidental movements, but rather through insight and purposeful action\footcite{koehler1921mentality}. It was Köhler's contention that intelligence not only consists of the mechanical repetition of previously learned actions but also that, for appropriate solution, a true comprehension of the problem must exist in some more or less definite manner in order to allow flexible, inventive solutions\footcite{koehler1921mentality}. His famous experiments with the chimpanzee Sultan, where he observed the ways in which Sultan learned to stack boxes so as to reach a banana, showed that such behavior was not trial-and-error learning but rather the reflection of Sultan's capacity for cognitive problem-solving\footcite{koehler1921mentality}. Sultan's stop to consider the problem and then to act with deliberation showed that he had cognized the problem as well as possible solutions to that problem, and thus considered the kinds of thinking through problems typical of humans.

Another critical dimension of Köhler's work was the use of roundabout methods or detours in testing intelligence\footcite{koehler1927detours}. He observed that if the way was not obstructed, a higher animal would, by nature, tend to proceed along a straight line towards any desired goal. But when various obstructions were in the way, the necessity of an indirect route presented a different and far superior plane of complication. The ability to surmount such detours intelligently was considered by Köhler as the higher-plane psychical ability\footcite{thorndike1898animal}. In one of the most straightforward experiments, chimpanzees were required to make a detour around an obstacle to reach food that was visible but could not be directly accessed. When the obstacle, the goal, and the potential detours were all within view, chimpanzees could typically solve the problem with ease. The ability to mentally plan a pathway through a complex environment is a crucial aspect of animal intelligence.

Köhler's observations did not stop at successful attempts. He observed that each chimpanzee varied in solution: some methods worked much more satisfactorily than others. This one, Chica, automatically went to a roof and attempted to reach the target from above, usually showing so much dexterity and problem-solving ability that Köhler was impressed. The other chimpanzees, like Grande, were more cautious in their approach but could learn through observation and imitation of successful methods devised by their peers. These experiments really underlined the individuality of animal intelligences and their diversity in solving problems\footcite{koehler1921mentality}.

The roundabout methods underlined another fascinating phenomenon: whereas chimpanzees were doing quite well when they had the hurdles in front of them, which they needed to overcome, once the invisible main part of a problem was not visible to them, their ability to find the solution dropped significantly. Köhler tried testing this by putting food out of sight or in experimental apparatus in which only part of the solution could be seen. The chimpanzees in these situations would need to rely on memory and/or prior experience. For example, Sultan was able to solve such complicated puzzles by drawing from past knowledge, meaning that intelligence is an issue not only of direct observation but also of the capacity to integrate past experiences into present problem-solving\footcite{koehler1921mentality}.

Köhler contrasted these findings with other species' reactions to similar problems. In one experiment with a dog, the animal was put in a blind alley with visible food on the other side of the fence. It would first attempt to go straight for its food, find itself impeded by the fence, and after some hesitation in its movements, it learned it had only to run around the fence\footcite{morgan1894animal}. But when the food was placed much closer to the fence, the dog became confused, pushing against the fence rather than using the detour it had previously learned. This he attributed to the fixation of the animal on the immediate proximity of the objective, overriding the understanding of the animal for the necessity of making a detour. The behavior brought out the complexity in how various animals process obstacles and that proximity to the goal actually hindered problem-solving even in intelligent animals.

Other interesting insights brought about by the experiments of Köhler referred to the way children approached roundabout methods. In one such instance, similar to the problem the chimpanzees were faced with, a toddler encountered an issue when food was on the other side of a barrier. The child started trying to push through the barrier but, having stopped, suddenly glanced around and made his happy discovery of walking around the obstacle to accomplish his task\footcite{thorndike1898animal}. The joyful reaction marked this instant of insight, when suddenly one perceives the solution to the problem at hand and then rapidly acts on it. This point reminded him of the kinds of behaviours the chimpanzee had given when in similar predicaments and strengthened the view of both there being the facility within chimpanzees and humans for solving problems through insight\footcite{koehler1921mentality}.

In contrast, when taking these results onto lower animals such as hens, Köhler felt these animals gave a very different response. Many hens, presented with similar roundabout problems, were prone to panic and not to find the solution. Instead of reasoning out the detour, they would have pecked back and forth along the obstruction in utter bewilderment\footcite{koehler1927detours}. By a long process of time or by mere chance few hens might find the right route through. Such happened, however, as in marked contrast with the method followed by the chimpanzee, and furnished thereby a good indication of the varieties between species in solving such problems\footcite{thorndike1898animal}. Köhler also made a provisional accounting for solution of these problems by chance particularly in the lower animals. Some would stumble upon the solution by chance; the problem then becomes one of how to separate the occurrence of true intelligence from random success. Köhler believed that the distinction rested on the smoothness and flow of the action: a true solution is spatially and temporally coherent, he argued, whereas chance solutions are discontinuous with various false starts and stops en route\footcite{koehler1927detours}. But this is an important distinction to make in understanding the difference between learned behavior and true insight\footcite{thorndike1898animal}.

According to Köhler, the implications of these experiments go far beyond simple problem-solving and form a window to the mental processes that govern complex behavior. The ability to solve these problems consistently—using insight and spatial awareness—suggests that their intelligence is not solely a product of trial and error\footcite{koehler1921mentality}. That points to a more profound, highly developed mental ability—one that enables the ant to predict the outcome, recognize cause-and-effect relations, and so behave accordingly. These findings challenge earlier conceptions that thought processes of this nature are the exclusive domain of humans and propose that intelligence in animals may be far more diffuse than was hitherto considered\footcite{koehler1927detours}.

The circumventing ways also raise a larger question about how we judge intelligence in non-human species. The experiments conducted by Köhler make it clear that intelligence is not a unitary characteristic but rather one which expresses itself differently in different species and in different contexts. In the case of certain animals, such as chimpanzees, intelligence entails flexibility, adaptability, and the ability for abstract thinking. In other cases, hens or even dogs of some kind, problem-solving may be more related to trial and error or even chance\footcite{thorndike1898animal}. This variability underlines the need for more differential approaches to the study of animal cognition, with consideration of the differences between species and their specific cognitive capabilities\footcite{morgan1894animal}.

Conclusion: Köhler's experimentation with roundabout methods and detours speaks volumes about animal intelligence. Experiments by this great scientist have proven that higher animals, such as chimpanzees, are capable of behaving intelligently in complicated environments by insightful and flexible solution-finding, not associated with trial-and-error methods. By comparing this behavior to that of the lower animals, Köhler was able to bring out the subtlety in the thought processes involved with problem-solving\footcite{koehler1921mentality}. His research continues to be a cornerstone in the study of intelligence, challenging all researchers to further refine their understanding of what it really means to be intelligent.

\newpage





\section{Reflections on Animal and Human Intelligence}

The nature of intelligence, whether human or animal, has always been an issue that is debated scientifically, philosophically, and culturally. Whereas problem-solving, flexibility, and the ability to learn have traditionally been held as the hallmarks of intelligence, the concept of intelligence itself varies widely in different areas and cultures. Each of them—philosophers, psychologists, biologists, and even indigenous societies—offers a very unique perspective that challenges and enriches our notion of what it is to be intelligent\footcite{gardner1999intelligence}. This section presents the current debate on diversity in perspectives: animal intelligence as adaptive and indigenous/cultural views as presenting a wider, more holistic view of intelligence.

The view of intelligence from a cultural and philosophical perspective looks quite different than that from the Western emphasis on cognitive problem-solving abilities. Intelligence in the West is often associated with reason, abstract conceptualization, and technological development\footcite{nisbett2003geography}. Intelligence is measured through standardized tests that emphasize logical reasoning, memory, and analytical thinking. While useful in certain contexts, this approach often overlooks more nuanced and ecological forms of intelligence both in nature and in non-Western societies\footcite{sternberg2004wisdom}. Philosophically, such luminaries as Aristotle and Descartes viewed intelligence as that which defines and differentiates humans from animals\footcite{aristotle1986nicomachean}. As typified by the famous dictum "I think, therefore I am" from Descartes, there was a belief that rational thought was the highest form of intelligence\footcite{descartes1996discourse}. This is a human-centered view that, however, has increasingly come under challenge from modern cognitive scientists and ethologists through the demonstration of sophisticated behaviors from many animals that suggest intelligence. Instances are tool use, complex communication, and self-awareness in animals that question the conventional notion of intelligence as something uniquely human\footcite{shettleworth2010cognition}. Indigenous cultures, on their part, often approach intelligence from the point of view of harmony with the natural world. In many indigenous cultures, the concept of intelligence is not solely associated with the cognitive capability of an individual but rather a relation to one's surroundings, other forms of life, and even the cosmos itself\footcite{berkes2012sacred}. This is, of course, a far cry from the Western emphasis on individualism and competitiveness in defining intelligence.

The ability of animals to solve problems, use tools, and adapt to their environments is considered an important measure of intelligence\footcite{emery2004cognitive}. Animal intelligence is an instinctive tendency for survival, learning from experience, and adjustment of behavior with every alteration in the environment. 
Indeed, flexibility is a characteristic feature that can be reflected in animal problem-solving strategies, 
proving thereby that intelligence can never be an end but a process of adaptation\footcite{dewaal2016are}. 
Probably one of the most salient manifestations of animal intelligence is the use of tools. The following examples are well-known: chimpanzees using sticks to draw out termites from mounds and crows being documented while bending wires into hooks that enable them to get food from otherwise hard-to-reach places\footcite{hunt2000crows}. These behaviors do indeed point to a depth of cognitive sophistication in terms of planning, foresight, and object manipulation in creative ways\footcite{emery2004cognitive}. These are not instinctive behaviors but, rather, require the animal to have an understanding of the properties of the tools they are using and the problems they are trying to solve. This adaptive intelligence does not come solely from primates and birds. Dolphins, elephants, and even octopuses have been observed using tools or solving complex problems, indicating that intelligence can take many forms across the animal kingdom\footcite{janik2006communication}. What these examples illustrate is that intelligence in animals generally involves a significant level of flexibility and adaptability. Switching strategies in the presence of novel challenges, learning from experience, and transferring knowledge across situations—just to name a few—are among the hallmarks of adaptive intelligence\footcite{dewaal2016are}.

Perhaps one of the most unusual intelligences of animals can be found in the hunting and navigational abilities of falcons. They also demonstrate intelligence sculpted by their specific ecological niche. With a very clear vision, they can see their prey from a distance; with streamlined bodies, they can fly with precision in complex maneuvers at hunting. For example, falcons, like the peregrine falcon, can dive at more than 240 miles per hour in their dive to catch prey right out of mid-air, an action that greatly requires exceptional spatial awareness, timing, and control\footcite{tucker2000gliding}. Falcons also exercise a kind of problem-solving intelligence in their capacity to make very long migratory routes, many times over thousands of miles\footcite{wiltschko2003navigation}. The migrating falcons align themselves by combining the optic topography together with the Earth's magnetic field and even the position of the sun\footcite{wiltschko2003navigation}. The ability to navigate across diverse and often challenging environments speaks to a highly specialized form of intelligence, one that is finely tuned to their role in being top predators in the avian world.

That being said, the intelligence of falcons points to yet an interesting contrast between specialized and general intelligences. Outstanding in their particular domain, yet the intelligence of the falcons is held to be instinctive, rather than extending into more general tasks. In more particular depth, this discussion shall show us whether intelligence in animals should be viewed as one spectrum or not, ranging from specialized to general cognitive skills. Specialized intelligence refers to the building up of specialized capabilities by animals to enable them to thrive well in their ecological niches. In this respect, over millions of years, falcons have acquired hunting skills and have turned out to be efficient predators\footcite{tucker2000gliding}. Similar is the specialized intelligence expressed by other species, such as ants, who work in a very coordinated collective, construct complex colonies, and even leave behind them a trail of chemicals to communicate\footcite{wilson2000sociobiology}. Such forms of intelligence are highly effective in the contexts in which these animals find themselves but, at the same time, do not easily generalize to other environments or tasks\footcite{shettleworth2010cognition}. General intelligence, however, is typically characterized by knowledge application across a wide range of situations. Humans have a general intelligence since we can think abstractly, solve problems in many contexts, and adapt to new as well as changing circumstances. Some animals, however, do exhibit types of general intelligence: notably, some primates do\footcite{dewaal2016are}. Thus, for example, chimpanzees employ tools not in just one context but in many, and they can learn new problem-solving strategies by observing others. This brings up some very important questions about the distinction between specialized and general intelligence—the way in which we measure intelligences across species. It suggests that intelligence is not a single, unitary trait; rather, it seems to be a collection of abilities that vary depending on the animal's ecological and social needs\footcite{emery2004cognitive}. Diverse forms of intelligence within the animal kingdom thwart the view of intelligence as unidimensional—linearly ranked or measurable\footcite{dewaal2016are}.

Cultural perspectives, in addition, open up dimensions beyond what narrowness Western conceptualizations of intelligence have grappled with. For many indigenous cultures, intelligence is a form of interconnectivity with Nature. Indigenously, the concept many times is more ecologically biased, focusing less on cognitive abilities and stressing a perspective of harmony with the environment, whereby one survives sustainably within natural ecosystems\footcite{berkes2012sacred}. Among the Quechua of Peru, for example, intelligence is looked at in terms of "runa yachay"—the knowledge of living in harmony with the earth\footcite{berkes2012sacred}. A form of intelligence which would include awareness of natural cycles, reading signs in the environment, and wisdom in decision-making for the good of the community and future generations. This holistic view of intelligence runs in opposition to the individualistic, competition-driven models that dominate Western thought\footcite{sternberg2004wisdom}.

Indigeneity and the concept of intelligence have been related to the notion of Gaia—or the supposition that the Earth is a sentient organism with its intelligence—only too often\footcite{lovelock2000gaia}. First coined into popular usage by a scientist named James Lovelock in something called the Gaia hypothesis, this supposition infers that everything is connected and that intelligence does not reside in discrete organisms but resides in the systems through which life is sustained. This is the kind of view that has long been cherished by many indigenous cultures—viewing intelligence as something delivered not from within the brain but as a distributed property of the earth and its ecosystems\footcite{lovelock2000gaia}. For example, the Shipibo people of the Amazon rainforest believe that one derives intelligence from listening to the earth, knowing signs and signals provided by nature\footcite{berkes2012sacred}. They think of plants, animals, rivers, and mountains as beings that possess their own ways of intelligent communication and knowledge\footcite{berkes2012sacred}. What this might seem to be suggesting, therefore, is that there is a real case for such a hierarchical view in regard to intelligence—not of placing humans at the top of some cognitive ladder but rather that real intelligence is a way of living in balance with the natural environment\footcite{lovelock2000gaia}.

The Gaia hypothesis resonates with these indigenous views and postulates that life on earth represents a complex, self-sustaining system that maintains conditions favorable for its survival\footcite{lovelock2000gaia}. Through this prism, intelligence would not be confined to human entities; rather, it would represent a planetary feature per se. Conceiving intelligence in this way removes the focal point from the realm of individualized cognition to broader-scale ecological processes of life maintenance\footcite{lovelock2000gaia}. In other words, reflection on animal and human intelligence alike is in need of extension of our definitions and consideration of the diversified fashions in which intelligence manifests across species and cultures. From the adaptive problem-solving abilities in animals to the holistic, interconnected views on intelligence among indigenous cultures, it becomes clear that there cannot be one particular set of cognitive skills to which intelligence might be reduced. An expanded concept of intelligence enables us to better appreciate the complexity of life on Earth and the various ways in which living organisms interact with their ambient environment\footcite{dewaal2016are}.

\newpage



\section{Reevaluating Intelligence: A Broader View}

One of the historical challenges faced by philosophy, psychology, and biology is the definition of intelligence. Traditionally, intelligence has been related to cognitive abilities such as reasoning, remembering, problem-solving, and learning\footcite{sternberg2004wisdom}. In most Western academics and culture, these cognitive skills have been portrayed as the main indicators of intelligence, especially those related to logic and abstract thinking\footcite{gardner1999intelligence}. But since then, this has come into question in light of studies in animal intelligences and cross-cultural perspectives which show a much more wide-reaching idea of what it is to be intelligent. Intelligence, it would seem, cannot be fully encapsulated in a narrow set of cognitive skills but also envelops adaptability, social and ecological awareness, even emotional intelligence\footcite{goleman1995emotional}.

In the Western tradition, the concept of intelligence has been cast traditionally within a hierarchical lens, with human beings considered the pinnacle of cognitive abilities\footcite{aristotle1986nicomachean}. Through philosophers like Descartes and Kant came arguments that human rationality was what separated us from animals, and thinking abstractly, language development, and the ability to reason logically were higher orders of intelligence\footcite{descartes1996discourse}. This perspective has typified most Western scientific thought on the development of standardized intelligence tests that quantify verbal reasoning, mathematical, and spatial awareness among other traits\footcite{sternberg2004wisdom}. While these tests have been useful in certain contexts, particularly in educational settings, there are considerable limitations. They often fail to represent the breadth of human intelligence and altogether neglect other forms of intelligence that are crucial to survive and thrive in different ecological settings\footcite{gardner1999intelligence}.

Indigenous views on intelligence, on the other hand, are profound and holistic. In many indigenous cultures, intelligence is viewed as inseparable from the environment and community, not confined to an attribute of the individual\footcite{berkes2012sacred}. Intelligence, in this regard, can be more or less articulated with an understanding of nature and the way one is able to survive harmoniously with the environment\footcite{berkes2012sacred}. Rather than emphasize abstract reasoning or problem-solving abilities, such cultures emphasize ecological intelligence—that is, the ability to cue environmental events, understand seasonal patterns, and use resources in a sustainable manner. It is knowledge in the form of stories, traditions, and practices passed down through generations that enables communities to survive and even thrive\footcite{berkes2012sacred}. Probably one of the most striking examples of ecological intelligence can be found in indigenous peoples of the Amazon rainforest, including the Yanomami and Shipibo. These communities have built up profound knowledge concerning medicinal plants, animal behaviors, and seasonal forest cycles\footcite{berkes2012sacred}. The knowledge is not only functional but also spiritual; most indigenous peoples view the relationship between themselves and the environment with a deep mutual respect\footcite{berkes2012sacred}. It is not about individual accomplishment or competition in the context of intelligence but living harmoniously with nature, maintaining equilibrium and sustainability for the bigger system.

Expanding the definition of intelligence to involve ecological and social domains disrupts the traditional hierarchy that has long shaped Western thought. Intelligence is generally ranked according to a linear scale with humans at the top, especially those from technologically advanced countries, and other animals and less industrialized cultures ranked lower\footcite{nisbett2003geography}. However, with the emergence of research that understood both the complexity of animal behaviors and the sophisticated knowledge systems of indigenous peoples, such hierarchies started to be criticized\footcite{dewaal2016are}. Animals such as dolphins, elephants, and primates have exhibited problem-solving abilities, social intelligence, and emotional awareness comparable to, if not matching, those of humans\footcite{dewaal2016are}. In the same vein, indigenous people in some of the least hospitable parts of the world have devised ways to survive that show intelligences not only adaptive but long-term sustainable\footcite{berkes2012sacred}.

This reassessment of intelligence requires us to shift away from rigid hierarchies and toward an inclusive understanding of intelligence that recognizes other forms of intelligence as worthy\footcite{goleman1995emotional}. While reasoning and memory are important cognitive abilities, they do not represent the sole markers of intelligence. Of equal importance is emotional intelligence, social intelligence, and ecological intelligence—particularly in today's world, when environmental concern is fast developing into a global issue\footcite{lovelock2000gaia}. Holistic conceptualizations of intelligence, taking into consideration the interconnectivity of individuals with their communities and environments, offer more inclusively expansive ideas of what it means to be intelligent.

The critique of traditional intelligence hierarchies has particular importance with regard to the narrow value of standardized tests. Whereas such tests may be of some utility in a limited number of situations, they often reflect a narrow conception of intelligence that is partial to particular cultural and educational backgrounds\footcite{sternberg2004wisdom}. They do not take into consideration the range of skills and abilities deployed by people in various societies and environments\footcite{gardner1999intelligence}. A person brought up in a rural or indigenous community might have deep knowledge about plants, animals, and natural cycles—skills that are critical for survival but would not be reflected in conventional intelligence tests\footcite{berkes2012sacred}. While the latter (emotional intelligence—the ability to sense and handle one's emotions and the feelings of others—is rarely given as much importance as cognitive skills, it is equally vital for personal well-being and social cohesion\footcite{goleman1995emotional}.

It is among indigenous knowledge systems that the most significant lessons can be learned about the wider meaning of intelligence. Perhaps one of the most striking notions related to indigenous approaches is the so-called Gaia hypothesis of James Lovelock\footcite{lovelock2000gaia}. It postulates that Earth is a self-regulating living body and all forms of life interconnect to provide a balance in support of life. Such a view has long been intuitively recognized by Indigenous peoples who, from this ecological awareness, have developed knowledge systems. There, intelligence is not thought of as a property of the individual in isolation but rather as something that emerges from relations between people, animals, plants, and the land itself\footcite{lovelock2000gaia}. This holistic notion of intelligence contrasts with the Western tendency to think of intelligence as a value-neutral, even quantifiable capacity of individuals that can be measured and ranked\footcite{sternberg2004wisdom}. It focuses on wisdom, balance, and sustainability instead.

In other words, intelligence should be viewed as a multidimensional construct rather than a single entity of cognitive ability\footcite{gardner1999intelligence}. Although traditional Western perspectives on intelligence have focused on abstract reasoning, memory, and problem-solving capabilities, there is an increasing awareness that emotional, social, and ecological aspects also come under the definition of intelligence\footcite{goleman1995emotional}. It is within indigenous knowledge systems and studies into animal intelligence that such broader forms of intelligence become clear, challenging traditional hierarchies that have until recently defined our understanding of what it means to be intelligent\footcite{dewaal2016are}. By opening the definition of intelligence toward these diverse aspects, a more inclusive holistic understanding of the many ways in which living beings navigate and thrive in respective environments becomes possible\footcite{berkes2012sacred}.

\newpage

\section{Discussion: Where is Intelligence?}

For quite some time, philosophers, scientists, and scholars from various disciplines have debated where intelligence resides. Is intellect mainly a part of the brain and just a cognitive function, or does it extend beyond that—into emotional, instinctive, or spiritual realms? According to the traditional view, particularly in Western thought, intelligence is located in the mind and has its roots in cognitive processes such as reasoning, memory, and problem-solving skills\footcite{sternberg2004wisdom}. This has increasingly been set against the view that these three aspects of human consciousness are interrelated and that intelligence operates at a more diffuse and distributed level than had hitherto been supposed\footcite{gardner1999intelligence}.

In exploring this question, one might consider how the mind, body, and soul relate and how intelligence functions or manifests across these planes of existence. Cognitive intelligence concentrated in the head has often been presumed to be the major form of intelligence in modern contemporary society\footcite{sternberg2004wisdom}. This involves abstract reasoning, logical thinking, and problem solving. Equally important in terms of the capacity to deal with social settings and psychological well-being is emotional intelligence, which may relate to the understanding and handling of one's emotions and the emotions of others\footcite{goleman1995emotional}. Since emotions are felt through bodily responses—fluctuating heartbeats, breathing rates, and hormonal changes—emotional intelligence suggests that intelligence is not strictly in the brain but also in the body's potential to regulate and respond to emotional stimuli\footcite{goleman1995emotional}.

Besides cognitive and emotional intelligence, instinctive intelligence is another vital form, primarily related to survival and adaptation\footcite{dewaal2016are}. Predisposed intelligence is a large component of unconscious behavior, dictating lifesaving actions, such as fight-or-flight responses or parental care. It is a deep portion of intelligence within the body that functions without conscious thought, effectively preserving life and ensuring species survival. Animals predominantly utilize instinctive intelligence to navigate their environments, find food, and avoid danger. In humans, instinctive intelligence manifests as gut feelings or intuition, helping individuals make decisions without apparent logical explanation\footcite{dewaal2016are}. These layers of intelligence—cognitive, emotional, and instinctive—are not independent entities but deeply intertwined; thus, intelligence can be considered a multidimensional flow between mind, body, and soul\footcite{gardner1999intelligence}.

The ramifications of this wider understanding of intelligence will have strong impacts on future research, especially in animal cognition and human psychology. Traditional investigations into intelligence have long focused solely on cognitive abilities, using standardized tests and behavioral experiments to quantify problem-solving skills or memory retention\footcite{sternberg2004wisdom}. As our notion of intelligence expands, future research will need to consider the emotional and instinctive dimensions of intelligence. For example, research on animal cognition will need to evaluate how animals navigate social interactions and respond to emotional stimuli in their environments, alongside their problem-solving abilities. Similarly, human intelligence studies must give more attention to emotional self-regulation and instinctive decision-making rather than purely cognitive functions\footcite{dewaal2016are}.

Furthermore, an alternative conception of intelligence can inform the construction and interpretation of future studies. If intelligence were seen as a unitary concept encompassing cognitive, emotional, and instinctive components, the structure of experiments would need to account for this complexity. For example, studies might incorporate physiological measures like heart rate or hormonal levels, alongside cognitive assessments, to gain insight into how social and environmental factors impact intelligent behavior. An interdisciplinary approach linking neuroscience, psychology, biology, and philosophy may help to better understand the multilevel nature of intelligence\footcite{goleman1995emotional}.

This extended view of intelligence may also influence research into artificial intelligence (AI). Current AI systems are designed primarily to mimic human cognitive abilities, such as logical reasoning, data processing, and problem-solving. However, as we begin to understand the importance of emotional and instinctive intelligence, these aspects may become necessary additions to AI systems. This could lead to AI systems that better suit social situations, respond to emotional signals from others, and adapt to changing environments, similar to human and animal intelligences\footcite{lovelock2000gaia}.

\newpage
\section{Conclusion}

In conclusion, research into intelligence—whether in humans, animals, or machines—should be broadened toward a multidimensional framework. Traditional views that focus only on cognitive intelligence are insufficient to describe the full range of intelligent behavior displayed by humans and animals. Emotional intelligence, instinctive intelligence, and even spiritual or ecological intelligence all play crucial roles in how living beings interact with their environments and make decisions. These forms of intelligence are interconnected, suggesting that intelligence is not a singular trait but rather a network of abilities flowing from the mind, through the body, and into the soul.

Insights gained from the study of animal cognition, indigenous perspectives, and holistic approaches to intelligence challenge the hierarchical models that have dominated Western thought for centuries. Expanding our concept of intelligence to include emotional and ecological dimensions allows for a deeper understanding of the diversity in nature and opens new avenues for research in psychology, neuroscience, and artificial intelligence.

Ultimately, intelligence is a multifaceted concept that cannot be fully understood outside the context of human cognition, animal behavior, or, more recently, artificial intelligence. For human cognition, animal behavior, and AI alike, intelligence is better conceptualized as a complex adaptive process intricately tied to the environments in which it operates. As we continue to explore and redefine intelligence, we must recognize the richness and diversity of intelligent life on Earth.

\newpage



\printbibliography

\end{document}