\documentclass[12pt]{article}
\usepackage[utf8]{inputenc}
\usepackage{amsmath}
\usepackage{amssymb}
\usepackage{graphicx}
\usepackage{geometry}
\usepackage[backend=biber, style=authortitle, citestyle=verbose-trad1]{biblatex}
\geometry{a4paper, margin=1in}
\addbibresource{references.bib}


\title{Where is Intelligence?}
\author{Daniel Schellhorn}
\date{\today}

\begin{document}

\maketitle

\section{Introduction}

Intelligence is a mystery that has been studied by psychologists, biologists, philosophers, and every other available field for years. As much as human intelligence has been researched for centuries, the study of intelligence is increasingly becoming focused on understanding intelligence in animals. This is because one now recognizes that such cognitive abilities are not solely characteristic of humans but abound in different levels through various animal species, hence questioning the linearity of human and non-human minds.

Understanding the intelligence of humans and animals is important on many levels. First, it offers insight into the evolutionary origins of cognitive abilities and problem-solving, which could indicate a biological predisposition to intelligence. Second, understanding how animals solve complex problems in adapting to their natural environments assists scientists in refining their notion of intelligence beyond the human-centered model. The research is important, most specifically, to provide detailed information on the principles underlying learning and behavior which may also extend to other general areas of study in the fields of artificial intelligence, neuroscience, and behavioral ecology.

This seminar paper looks at the multidimensionality of intelligence based on Köhler's own work with animals, particularly in his experiments with the primate species. Köhler defines intelligence as the ability of an organism to solve a problem through insight, not by trials and errors; it is on this definition that this review is based. The findings presented by Köhler are compared with other research, for instance those done by Thorndike, to point out the differences between true intelligence and chance behavior. This paper, therefore, tries to go back to the general understanding of what constitutes intelligent behavior in animals and, by extension, humans through the critical analyses of various methodologies applied in the assessment of intelligence.

\newpage

\section{The Intelligence of Köhler}

Of particular consequence in this regard are the experiments with chimpanzees, in which Köhler has endeavored to work out a rather more characteristic approach to the understanding of intelligence. According to him, the true intelligence manifests itself not so much in the fact that an animal solves its problems as in the manner of solving these problems, that is, not through random or accidental movements, but rather through insight and purposeful action\footcite{koehler1917chimpanzees}. It was Köhler's contention that intelligence not only consists of the mechanical repetition of previously learned actions but also that, for appropriate solution, a true comprehension of the problem must exist in some more or less definite manner in order to allow flexible, inventive solutions\footcite{koehler1921mentality}. His famous experiments with the chimpanzee Sultan, where he observed the ways in which Sultan learned to stack boxes so as to reach a banana, showed that such behavior was not trial-and-error learning but rather the reflection of Sultan's capacity for cognitive problem-solving\footcite{koehler1921mentality}. Sultan's stop to consider the problem and then to act with deliberation showed that he had cognized the problem as well as possible solutions to that problem, and thus considered the kinds of thinking through problems typical of humans.

Another critical dimension of Köhler's work was the use of roundabout methods or detours in testing intelligence\footcite{koehler1927detours}. He observed that if the way was not obstructed, a higher animal would, by nature, tend to proceed along a straight line towards any desired goal. But when various obstructions were in the way, the necessity of an indirect route presented a different and far superior plane of complication. The ability to surmount such detours intelligently was considered by Köhler as the higher-plane psychical ability\footcite{thorndike1898animal}. In one of the most straightforward experiments, chimpanzees were required to make a detour around an obstacle to reach food that was visible but could not be directly accessed. When the obstacle, the goal, and the potential detours were all within view, chimpanzees could typically solve the problem with ease. The ability to mentally plan a pathway through a complex environment is a crucial aspect of animal intelligence.

Köhler's observations did not stop at successful attempts. He observed that each chimpanzee varied in solution: some methods worked much more satisfactorily than others. This one, Chica, automatically went to a roof and attempted to reach the target from above, usually showing so much dexterity and problem-solving ability that Köhler was impressed. The other chimpanzees, like Grande, were more cautious in their approach but could learn through observation and imitation of successful methods devised by their peers. These experiments really underlined the individuality of animal intelligences and their diversity in solving problems\footcite{koehler1921mentality}.

The roundabout methods underlined another fascinating phenomenon: whereas chimpanzees were doing quite well when they had the hurdles in front of them, which they needed to overcome, once the invisible main part of a problem was not visible to them, their ability to find the solution dropped significantly. Köhler tried testing this by putting food out of sight or in experimental apparatus in which only part of the solution could be seen. The chimpanzees in these situations would need to rely on memory and/or prior experience. For example, Sultan was able to solve such complicated puzzles by drawing from past knowledge, meaning that intelligence is an issue not only of direct observation but also of the capacity to integrate past experiences into present problem-solving\footcite{koehler1921mentality}.

Köhler contrasted these findings with other species' reactions to similar problems. In one experiment with a dog, the animal was put in a blind alley with visible food on the other side of the fence. It would first attempt to go straight for its food, find itself impeded by the fence, and after some hesitation in its movements, it learned it had only to run around the fence\footcite{morgan1894animal}. But when the food was placed much closer to the fence, the dog became confused, pushing against the fence rather than using the detour it had previously learned. This he attributed to the fixation of the animal on the immediate proximity of the objective, overriding the understanding of the animal for the necessity of making a detour. The behavior brought out the complexity in how various animals process obstacles and that proximity to the goal actually hindered problem-solving even in intelligent animals.

Other interesting insights brought about by the experiments of Köhler referred to the way children approached roundabout methods. In one such instance, similar to the problem the chimpanzees were faced with, a toddler encountered an issue when food was on the other side of a barrier. The child started trying to push through the barrier but, having stopped, suddenly glanced around and made his happy discovery of walking around the obstacle to accomplish his task\footcite{thorndike1898animal}. The joyful reaction marked this instant of insight, when suddenly one perceives the solution to the problem at hand and then rapidly acts on it. This point reminded him of the kinds of behaviours the chimpanzee had given when in similar predicaments and strengthened the view of both there being the facility within chimpanzees and humans for solving problems through insight\footcite{koehler1921mentality}.

In contrast, when taking these results onto lower animals such as hens, Köhler felt these animals gave a very different response. Many hens, presented with similar roundabout problems, were prone to panic and not to find the solution. Instead of reasoning out the detour, they would have pecked back and forth along the obstruction in utter bewilderment\footcite{koehler1927detours}. By a long process of time or by mere chance few hens might find the right route through. Such happened, however, as in marked contrast with the method followed by the chimpanzee, and furnished thereby a good indication of the varieties between species in solving such problems\footcite{thorndike1898animal}. Köhler also made a provisional accounting for solution of these problems by chance particularly in the lower animals. Some would stumble upon the solution by chance; the problem then becomes one of how to separate the occurrence of true intelligence from random success. Köhler believed that the distinction rested on the smoothness and flow of the action: a true solution is spatially and temporally coherent, he argued, whereas chance solutions are discontinuous with various false starts and stops en route\footcite{koehler1927detours}. But this is an important distinction to make in understanding the difference between learned behavior and true insight\footcite{thorndike1898animal}.

According to Köhler, the implications of these experiments go far beyond simple problem-solving and form a window to the mental processes that govern complex behavior. The ability to solve these problems consistently—using insight and spatial awareness—suggests that their intelligence is not solely a product of trial and error\footcite{koehler1921mentality}. That points to a more profound, highly developed mental ability—one that enables the ant to predict the outcome, recognize cause-and-effect relations, and so behave accordingly. These findings challenge earlier conceptions that thought processes of this nature are the exclusive domain of humans and propose that intelligence in animals may be far more diffuse than was hitherto considered\footcite{koehler1927detours}.

The circumventing ways also raise a larger question about how we judge intelligence in non-human species. The experiments conducted by Köhler make it clear that intelligence is not a unitary characteristic but rather one which expresses itself differently in different species and in different contexts. In the case of certain animals, such as chimpanzees, intelligence entails flexibility, adaptability, and the ability for abstract thinking. In other cases, hens or even dogs of some kind, problem-solving may be more related to trial and error or even chance\footcite{thorndike1898animal}. This variability underlines the need for more differential approaches to the study of animal cognition, with consideration of the differences between species and their specific cognitive capabilities\footcite{morgan1894animal}.

Conclusion: Köhler's experimentation with roundabout methods and detours speaks volumes about animal intelligence. Experiments by this great scientist have proven that higher animals, such as chimpanzees, are capable of behaving intelligently in complicated environments by insightful and flexible solution-finding, not associated with trial-and-error methods. By comparing this behavior to that of the lower animals, Köhler was able to bring out the subtlety in the thought processes involved with problem-solving\footcite{koehler1921mentality}. His research continues to be a cornerstone in the study of intelligence, challenging all researchers to further refine their understanding of what it really means to be intelligent.

\newpage

\printbibliography

\end{document}