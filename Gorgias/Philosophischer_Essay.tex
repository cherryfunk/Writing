
\documentclass[12pt,a4paper]{article}
\usepackage[utf8]{inputenc}
\usepackage[T1]{fontenc}
\usepackage[ngerman]{babel}
\usepackage{biblatex}
\addbibresource{bibliography.bib}

\title{Was ist für wen gut?: Eine Analyse des Guten in Platons Gorgias}
\author{Daniel Schellhorn}
\date{\today}

\begin{document}

\maketitle

\section{Darstellung des Problems}
Sokrates' Behauptung im Dialog "Gorgias", dass es schlimmer sei, Unrecht zu tun als Unrecht zu erleiden, wirkt in der Tat kontraintuitiv, insbesondere in einer Gesellschaft, die oft denjenigen, der leidet, ins Zentrum stellt. Diese Sichtweise scheint das Opfer zu idealisieren und den Täter zu vernachlässigen, indem sie die physischen und psychischen Konsequenzen des Leidens hervorhebt. Doch Sokrates dreht den Spieß um und legt den Fokus auf die moralische Integrität des Täters.

Sokrates argumentiert, dass die Schändlichkeit des Unrecht-Tuns nicht in seinen äußeren Manifestationen wie Schande oder Schmerz liegt, sondern in der Verfärbung der Seele selbst. Die Seele, verstanden als der Sitz des moralischen und rationalen Lebens, wird durch Ungerechtigkeit verdorben und verliert ihre Harmonie und Gesundheit. Für Sokrates ist das höchste Gut, was Platon als das "Gute an sich" bezeichnet, nicht materieller Gewinn oder sinnliches Vergnügen, sondern die Gerechtigkeit, Weisheit, Tapferkeit und Mäßigung – die vier Kardinaltugenden Platons.

Diese Tugenden sind Selbstzwecke; sie sind wertvoll an sich und nicht wegen irgendeines Nutzens oder Vergnügens, das sie möglicherweise bringen könnten. Daher ist das moralisch Gute intrinsisch und nicht extrinsisch wertvoll. Sokrates würde behaupten, dass eine Person, die Unrecht tut, die Tugend und damit das Gute an sich beschädigt. Dies führt zu einer Störung der Ordnung der Seele, was schlimmer ist als jede physische oder emotionale Wunde, die man erleiden könnte.

Darüber hinaus hebt die von Sokrates aufgestellte Analogie zwischen Lust und dem Guten die Wichtigkeit der Absicht hinter unseren Handlungen hervor. Lust, die aus ungerechten Handlungen entsteht, ist, wie Sokrates argumentiert, ein falsches Gut; sie ist flüchtig und oft begleitet von einem späteren Leid. Im Gegensatz dazu ist die Freude, die aus gerechten Handlungen entsteht, nachhaltig und trägt zur Harmonie der Seele bei. Hierbei ist das Gute, das mit der Tugend verbunden ist, nicht nur nützlich, sondern schön, weil es die Seele zur Wahrheit, zur Ordnung und zur Ausgeglichenheit führt.

Die Idee, dass die Gerechtigkeit an sich genossen werden sollte, reflektiert die tiefere platonische Philosophie, dass das Gute in der Welt der Ideen existiert, unabhängig von unserer physischen Realität. Tugendhaftes Handeln führt uns näher an diese Ideale heran und ermöglicht es der Seele, in ihrer wahren Natur zu leben. Dies ist das wahre Glück oder Eudaimonia in der griechischen Philosophie – nicht ein durch Vergnügen bedingter Zustand, sondern das Ergebnis einer wohlgeordneten und tugendhaften Seele. Diese Art von Seele, soll also im Endeffekt die "bessere" Seele sein.

Aber warum soll denn nun diese tugendhafte Seele, die bessere sein? D.h. sie ist vielleicht für die Gesamheit des Kosmos die bessere, aber wenn sie leidet, wäre es dann für sie selbst nicht einfach persönlich besser mehr Lust und weniger Leid zu empfinden? Warum soll denn das kollektive und universelle Gute das selbe sein, was für diese individuelle Seele Gut ist?

Die Frage, warum eine tugendhafte Seele besser sein soll, berührt den Kern der platonischen Ethik und Metaphysik. Platons Antwort, durch Sokrates im Dialog "Gorgias" und anderen Werken artikuliert, stützt sich auf die Annahme, dass es eine objektive Ordnung des Guten gibt, die über die subjektiven Wünsche des Einzelnen hinausgeht.

Die platonische Sicht sieht die Seele als das wahre Selbst, das durch die Praxis der Tugend zu ihrer höchsten Form gelangt. Tugend in diesem Sinne ist nicht nur ein moralischer Zustand, sondern eine Art und Weise des Seins, die mit der Idee des Guten im Einklang steht. Platons Idee des Guten geht über das Materielle hinaus und umfasst das Wahre und das Schöne. Es ist die höchste Idee, die Quelle der Erkenntnis und der Realität selbst.

Die tugendhafte Seele ist also besser, weil sie in Harmonie mit dieser universellen Ordnung lebt. Dies führt zu mehreren Schlüsselpunkten:

1. Objektive Wahrheit über subjektives Empfinden: Platon argumentiert, dass was wir als individuelles Vergnügen oder Leid empfinden, oft trügerisch ist. Wahre Zufriedenheit und dauerhaftes Glück sind nur zu erreichen, wenn wir in Übereinstimmung mit der objektiven Wahrheit leben. Dies bedeutet, dass die kurzfristigen Freuden, die aus ungerechten Handlungen resultieren, uns von der Wahrheit entfernen und daher letztendlich zu einem weniger erfüllten Leben führen.

2. Die Gesundheit der Seele: Wie ein Arzt, der das körperliche Wohl über das momentane Vergnügen des Patienten stellt, betont Sokrates die Bedeutung der Gesundheit der Seele über die flüchtigen Vergnügungen des Körpers. Eine tugendhafte Seele ist eine gesunde Seele, und eine gesunde Seele ist für das individuelle Wohlbefinden unerlässlich.

3. Eudaimonia – Das wahre Glück: Im Gegensatz zu einer flüchtigen Lust (Hedone), stellt Eudaimonia ein tiefes und beständiges Wohlbefinden dar, das aus einem tugendhaften Leben entsteht. Dieses Konzept von Glück ist umfassender als das, was Lust allein bieten kann. Es ist das Glück, das aus einem Leben der Tugend und der Erfüllung kommt, was wiederum die harmonische Ordnung des Kosmos widerspiegelt.

4. Das Gute als kollektives Ziel: Sokrates und Platon würden argumentieren, dass das, was für das Individuum gut ist, letztlich auch für die Gemeinschaft gut ist. Ein gerechtes Individuum trägt zum Wohl der Gemeinschaft bei, da Gerechtigkeit zu harmonischen Beziehungen und zum Wohl der Stadt oder Polis führt. Dies spiegelt sich in Platons späterem Werk "Die Republik" wider, wo die Gerechtigkeit innerhalb der Stadt mit der Gerechtigkeit in der individuellen Seele parallelisiert wird.

Für Platon ist also das individuelle Gute nicht von der universellen Ordnung getrennt, sondern ein Teil davon. Die individuelle Seele, die sich nach Tugend strebt, erfüllt ihre Rolle im Kosmos und erreicht dadurch ihr höchstes Gut. Es ist ein anspruchsvolles Ideal, das nicht nur verlangt, das Richtige zu tun, sondern auch aus den richtigen Gründen – nicht aus Angst vor Strafe oder dem Wunsch nach Belohnung, sondern weil es in Übereinstimmung mit der Wahrheit und dem Guten ist.

Doch immernoch ist im Endeffekt kein schlagendes Argument dafür gegeben, dass das Gute an sich, das Gute für die individuelle Seele ist. Es ist nur argumentiert worden, dass das Gute an sich, das Gute für die Gesamtheit des Kosmos ist. Aber warum soll das Gute an sich, das Gute für die Gesamtheit des Kosmos, das Gute für die individuelle Seele sein? Ist es wirklicher ein tieferes Glück für die individuelle Seele, wenn sie sich nach dem Guten an sich richtet, als wenn sie sich nach dem Guten für sich selbst richtet? Ist es nicht einfach ein tieferes Glück für die Gesamtheit des Kosmos, wenn die individuelle Seele sich nach dem Guten an sich richtet, als wenn sie sich nach dem Guten für sich selbst richtet? Wieso ist denn die Eudaimonia der individuellen Seele, wirklich das tiefere Glück? Die Seele ist doch glücklich wenn all ihre Bedürfnise befriedigt sind im Sinne der Maslowschen Pyramide. Und ist die tugendhafte Seele wirklich gesünder? Diese Seele erscheint gesund wenn all ihre Schmerzen getilgt sind, ganz in der Tradition des pathologisierenden Ansatz der westlichen Medizin und ihrer Definition der Gesundheit als genau dies, als der Abwesenheit von Schmerz und Dysfunktion.

\section{Hauptteil}
Hier entwickeln Sie den Hauptteil Ihres Essays. \cite{plato_kriton}

\section{Schluss}
Hier fassen Sie Ihre Argumente zusammen und ziehen ein Fazit. \cite{plato_menon}

\printbibliography

\end{document}
