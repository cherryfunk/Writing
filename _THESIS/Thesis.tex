\documentclass[12pt,a4paper]{article}
\usepackage{amsmath}
\usepackage{amssymb}
\usepackage[utf8]{inputenc}
\usepackage[T1]{fontenc}
\usepackage{csquotes}
\usepackage[backend=biber, style=authortitle, citestyle=verbose-trad1]{biblatex}
\usepackage{biblatex}
\usepackage{quiver}
\usepackage{tikz}
\usepackage{tikz-cd}
\addbibresource{references.bib}
\usetikzlibrary{positioning, shapes.geometric, arrows.meta}


\title{Sysmers as Systems: Entity-Relationship and categorical diagrams with syntactic structure applied to cognitive architectures and ontologies} 
\author{Daniel Schellhorn}
\begin{document}

\maketitle

\section*{Introduction}
“...We know only a very few — and, therefore, very precious—schemes whose unifying powers cross many realms.” — Marvin Minsky. \footcite[?]{Minsky1988}
\newline

The purpose of this thesis is to get a very tiny step in the right direction of creating such a realm-crossing scheme.  Where else to start but at the most general realm-crossing theories that exist? Category Theory and Systems Theory. However, the relations between categories and systems are not yet fully understood. The concept of a "sysmer" will be defined as a certain kind of system with additional requirements. This ywill rest on the basis of another newly defined concept called "sysm", a certain generalization of the mathematical concept of a category. All of this is done in order to understand and guide the way for a precise and formal mathematical description of (symbolic) systems in order to apply these in the fields of cognitive architectures and ontologies. To meet this goal, sysmers will be expressed as typed diagrams equipped with a syntactic structure and also represented as ER diagrams.
\section{Entity-Relationships (Data Science) \footcite{ElmasriNavathe2015}}

When referring to ER I always mean to refer to EER, as we will always use the enhanced version of the ER model, in this sense it is still the ER model but only a newer version $ER 2.0$, in other terms $EER = ER 2.0 \subseteq ER \supseteq ER 1.0 $. \footcite[107]{ElmasriNavathe2015}

\section{"Sysms" vs. Categories\footcite{MacLane1997} (Mathematics)}

\subsection{The Architecture of Mathematics}
We categorize the basic structures of mathematics\footcite[68]{Basieux2000} in the following categories (which are also categories in the mathematical sense):

...(sysmers in here)...

\subsection{Categories}

\subsection{Ologs}
``ologs, which serve as a bridge between mathematics and various conceptual landscapes.'' \footcite[24]{Spivak2014}

``A type is an abstract concept, a distinction the author has made. Each type is represented as a box containing a singular indefinite noun phrase.'' \footcite[25]{Spivak2014}

``when it comes to ologs, the word aspect simply means function. The domain A of the function \( f: A \to B \) is the thing we are measuring, and the codomain is the set of possible answers [...].'' \footcite[27]{Spivak2014}

The definition of a functional Olog is expanded by Spivak to the notion relational Olog in the following way:
\begin{quote}
In the context of the power-set monad, a morphism \( f: X \to Y \) between sets \( X \) and \( Y \), as objects in \( \textbf{Kl}(\mathcal{P}) \), becomes a binary relation on \( X \) and \( Y \) rather than a function. [...] An olog in which arrows correspond to mere binary relations rather than functions might 
be called a \textit{relational olog}. \footcite[447]{Spivak2014}
\end{quote}

\subsection{Schemas}
``a database schema is nothing but an olog in disguise. The difference is basically the readability requirements for ologs.'' \footcite[194]{Spivak2014}

``A category (as distinguished from a metacategory) will mean any interpretation of the category axioms within set theory'' \footcite[10]{MacLane1997}

Defining metacategories and categories as sysmers, a metacategory would be a sysmer schema and a category would be a sysmer instance:

...(put sysmers in here)...

Now if there are instances and schemas of sysmers, then there are most likely also operations on sysmers just like the ``functional requirements of the application. These consist of the userdefined operations (or transactions) that will be applied to the database'' \footcite[61]{ElmasriNavathe2015}

\subsection{Monads and Operads}

\subsection{Sysms}

\section{Sentences \footcite{Carnie2012} (Language)}

\section{Diagrams (Symbology\footcite{Jung1964})} 

\section{Cognitive Architectures (Cognitive Science)}
What magical trick makes us intelligent? The trick is that there is no trick. The power of intelligence stems from our vast diversity, not from any single, perfect principle. — Marvin Minsky. \footcite[308]{Minsky1988}

\[
	\resizebox{1.25\textwidth}{!}{
	\begin{tikzcd}[ampersand replacement=\&,sep=normal, cells={nodes={draw, minimum width=1cm, minimum height=0.5cm, anchor=center}}]
	\&\&\&\&\&\& Self \\
	\\
	\&\&\&\&\&\& Intellect \\
	\\
	\&\&\&\&\&\& {Slow Mind} \\
	\&\&\&\&\&\& Ego \\
	Nature \&\& Object \&\& Body \&\&\&\& Soul \&\& Subject \&\& Culture \\
	\&\&\&\&\&\& Socio \\
	\&\&\&\&\&\& {Fast Mind} \\
	\\
	\&\&\&\&\&\& Life \\
	\\
	\&\&\&\&\&\& Death
	\arrow["Motivatio", from=1-7, to=7-11]
	\arrow["Decisio"{description}, from=3-7, to=1-7]
	\arrow["Cognitio"{description}, from=5-7, to=3-7]
	\arrow["Emotio", from=5-7, to=7-9]
	\arrow["Emotio"{description}, from=6-7, to=7-9]
	\arrow["Vinculum"{description}, tail reversed, from=6-7, to=8-7]
	\arrow["Emergentio", from=7-1, to=7-3]
	\arrow["Observatio", squiggly, from=7-3, to=1-7]
	\arrow["Sensatio", from=7-3, to=7-5]
	\arrow["Preservatio"', squiggly, from=7-3, to=13-7]
	\arrow["Interpratatio", from=7-5, to=5-7]
	\arrow["Conditio"{description}, from=7-5, to=6-7]
	\arrow["Intuitio"', from=7-5, to=9-7]
	\arrow["Motivatio", from=7-9, to=7-11]
	\arrow["Expressio"{description}, from=7-9, to=8-7]
	\arrow["Creatio", from=7-11, to=7-13]
	\arrow["{{Sensatio`}}"{description}, from=8-7, to=7-5]
	\arrow["Emotio"', from=9-7, to=7-9]
	\arrow["Memorizatio"{description}, from=9-7, to=11-7]
	\arrow["Instictio"{description}, from=11-7, to=13-7]
	\arrow["Motivatio"', from=13-7, to=7-11]
\end{tikzcd}
}
\]

\section{Ontologies (AI)}

\section{Ontology\footcite{Platon2010} (Philosophy)}
relate: Cogn Arch. to Self Ontology of Plato (Instanz,Tugenden, Gute für jeweilige Instanz (Health, Wealth, Wisdom, Happiness,...), Vermögen, Aktivität, Function) and Psych. Types of Jung (Introversion, Extroversion, Thinking, Feeling, Sensing, Intuition) (sonnengleichnis pic in zotero)

In the Republic and Phaedro by Plato we hear about at least 3 types of ontology. 

The first distinguishes Gold, Silver and Bronze classes of the State.

Secondly these correspond with the instances of the Self; Soul, Body and Mind or Reason, Desire and Emotion.

The third level of ontology is the given by Allegories of the Line and of the Cave.

Is maybe the Allegory of the Cave the first description of a cognitive architecture?

Or is it also a convincing story of the cave, and to overcome and question that story is exactly the task of Philosophy? 
\section{"Sysmers" vs. Systems (Systems Theory)}

\subsection{Application in Category Theory}
We can apply sysmers in order to get "canonically ordered" diagrams of (meta-)categories and universal constructions like limits, as well as adjoints and natural transformations. This helps us to better generalize and abstract the underlying structures of these categorical constructions and also improves memory and intuition when working within the jungle of categories.


\section*{Conclusion}


\printbibliography

\end{document}
