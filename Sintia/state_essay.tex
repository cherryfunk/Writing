\documentclass{article}
\usepackage[utf8]{inputenc}
\usepackage{hyperref}
\usepackage[backend=bibtex,style=verbose-trad2]{biblatex}
\addbibresource{references.bib}

\title{The Effectiveness of Unsustainable Management of Human Capital Resources in Cameroon: Striving for More or Thriving with Less, It Is a Question of Choice}

\begin{document}

\maketitle

\section*{Introduction}

Cameroon, a nation with a diverse tapestry of cultures and languages, stands at a crossroads in managing its most vital resource: human capital. In this context, scarcity does not merely imply a lack of resources, but rather an underutilization and misalignment of existing skills and potentials. The nation grapples with the dilemma of whether to strive for more by investing in expanding capabilities or to thrive with less by optimizing and effectively deploying its existing human resources. Central to this predicament is the educational system, a legacy of the colonial era and the inefficient development policies, which have yet to fully align with the dynamic needs of the labor market\footcite{nya2016relationship}. This introduction seeks to unravel the complexities of this mismatch, examining how the development policy structures contribute to the underutilization of human capital in Cameroon and exploring pathways to reconcile this gap\footcite{kuepie2016determinants}. The discussion will delve into the intricacies of managing human capital amidst these challenges, shedding light on the nuanced interplay between education, skill development, and labor market demands within the Cameroonian context\footcite{ntamack2012education}\footcite{manguelle2021becoming}\footcite{ngwolefack2021contribution}.

\printbibliography

\end{document}
