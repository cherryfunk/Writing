\documentclass[12pt]{article}
\usepackage[utf8]{inputenc}
\usepackage{lipsum} % This package generates dummy text, you can remove it in your final document

\title{Comparative Analysis of Marxist Thought in the Context of Contemporary Africa}
\author{Sintia Kapleu}
\date{\today}

\begin{document}

\maketitle

\section{Introduction (Approximately 750 words)}
\subsection*{Brief Overview of Gramsci, Marx, and Lenin}
Introduce the primary theorists, highlighting their main contributions to Marxist thought.

\subsection*{Focus of the Paper}
Outline the purpose of the paper, emphasizing the comparative analysis of the Marxist trilogy and its relevance in contemporary Africa.

\subsection*{Historical Context}
Provide an overview of the socio-political climates during the lifetimes of Gramsci, Marx, Lenin, and the significant events and movements in Africa during the same period.

\subsection*{Comparative Contextual Analysis}
Compare and contrast the emergence of Marxist thoughts in Europe with African socio-political developments.

\section{Theoretical Framework (Approximately 1500 words)}
\subsection*{Core Ideas Analysis}
Examine Marx's historical materialism, Lenin's theories on imperialism and the vanguard party, Gramsci's concepts of hegemony and organic intellectuals.

\subsection*{Post-colonial Power and State Thoughts}
Incorporate Achille Mbembe's perspectives in relation to the Marxist trilogy.

\subsection*{The Trilogy's Impact on Africa}
Discuss how the theories of Marx, Lenin, and Gramsci influenced African movements.

\section{Between Continuity and Discontinuity (Approximately 1500 words)}
\subsection*{Capitalism Critique}
Assess Marx's, Lenin's, and Gramsci's views on capitalism and its cultural hegemony.

\subsection*{Mbembe's Perspective on Capitalism in Africa}
Analyze his thoughts on capitalism and postcolonial power dynamics.

\subsection*{State and Revolution}
Investigate the trilogy's and Mbembe's views on the state and revolution.

\section{What if Yesterday's Call Again (Approximately 1500 words)}
\subsection*{Intellectuals' Role in Power Dynamics}
Discuss Marx, Lenin, and Gramsci's perspectives on intellectuals and Mbembe's views on cultural production and leadership.

\subsection*{Soft Power Analysis}
Analyze the trilogy's and Mbembe's thoughts on soft power in the post-colony context.

\section{Neither In nor Out: A (Re)contextual Analysis (Approximately 1500 words)}
\subsection*{African Perspectives of Power, State, and Revolution}
Analyze how African perspectives align or diverge from Marx, Lenin, and Gramsci's theories.

\subsection*{Intersection with Afro-Movement}
Compare and contrast the trilogy's views with African movements.

\subsection*{Inclusivity or Exclusivity in Contemporary Challenges}
Evaluate the relevance of these theories in addressing current African socio-political issues.

\section{Conclusion (Approximately 750 words)}
\subsection*{Legacy and Impact}
Reflect on the long-term influence of these theorists on African socio-political thought.

\subsection*{Key Findings Summary}
Summarize the main findings of the paper.

\subsection*{Afro Pessimism and Optimism Movements}
Incorporate these movements in the final analysis, discussing their relevance to the theories examined.

\end{document}
