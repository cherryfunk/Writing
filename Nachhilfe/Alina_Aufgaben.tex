\documentclass{article}
\usepackage{amsmath}
\begin{document}

\section*{Algebraische Regeln und Aufgaben}

\subsection*{1. Potenzen eines Bruchs}
\textbf{Regel:} \( (\frac{a}{b})^m = \frac{a^m}{b^m} \)
\begin{itemize}
    \item \( \left(\frac{2}{3}\right)^5 \)
    \item \( \left(\frac{x}{x+1}\right)^3 \) für \( x \neq -1 \)
    \item \( \left(\frac{5a^2b}{3c}\right)^4 \)
\end{itemize}

\subsection*{2. Division von Produkten}
\textbf{Regel:} \( \frac{a \cdot b}{c} = a \cdot \frac{b}{c} \)
\begin{itemize}
    \item \( \frac{6x^2y}{2y} \)
    \item \( \frac{15p^3q^2}{5pq^4} \)
    \item \( \frac{a^2b^3c^2}{abc^4} \)
\end{itemize}

\subsection*{3. Division von Produkten (Erweitert)}
\textbf{Regel:} \( \frac{a \cdot b}{c \cdot d} = \frac{a}{c} \cdot \frac{b}{d} \)
\begin{itemize}
    \item \( \frac{4a^3b^2}{2ab^4} \)
    \item \( \frac{x^2y^3z}{xyz^2} \)
    \item \( \frac{16m^3n^5}{8mn^2} \)
\end{itemize}

\subsection*{4. Kommutativgesetz der Multiplikation}
\textbf{Regel:} \( a \cdot b = b \cdot a \)
\begin{itemize}
    \item Berechne \( 7 \cdot 8 \) und \( 8 \cdot 7 \).
    \item Vertausche die Reihenfolge der Faktoren in \( cab4abc \) und vereinfache.
    \item Wenn \( xy = 10 \) und \( yx = 10n \), bestimme \( n \).
\end{itemize}

\subsection*{5. Assoziativgesetz der Multiplikation}
\textbf{Regel:} \( a \cdot (b \cdot c) = (a \cdot b) \cdot c \)
\begin{itemize}
    \item Vergleiche \( (2 \cdot 3) \cdot 4 \) und \( 2 \cdot (3 \cdot 4) \).
    \item Vereinfache \( (ab) \cdot (cd) \) mit \( a = 2, b = 3, c = 4, d = 5 \).
    \item Zeige, dass \( (x \cdot y) \cdot z = x \cdot (y \cdot z) \) für \( x=2, y=3, z=4 \).
\end{itemize}

\subsection*{6. Addition gleicher Terme}
\textbf{Regel:} \( a + a = 2a \)
\begin{itemize}
    \item \( 5x + 3x \)
    \item \( -2a + (-3a) \)
    \item \( \frac{1}{2}y + \frac{3}{4}y \)
\end{itemize}

\subsection*{7. Division von Potenzen mit gleicher Basis}
\textbf{Regel:} \( \frac{a^m}{a^n} = a^{m-n} \)
\begin{itemize}
    \item \( \frac{2^8}{2^3} \)
    \item \( \frac{x^7}{x^2} \)
    \item \( \frac{5^6}{5^5} \)
\end{itemize}

\subsection*{8. Distributivgesetz}
\textbf{Regel:} \( a(b + c) = ab + ac \)
\begin{itemize}
    \item \( 7(x + y) \)
    \item \( a(2b - 3c + 4d) \)
    \item \( m(1/n + n^2 - \sqrt{n}) \)
\end{itemize}

\subsection*{9. Produkt von Potenzen mit gleicher Basis}
\textbf{Regel:} \( a^m \cdot a^n = a^{m+n} \)
\begin{itemize}
    \item \( 3^4 \cdot 3^5 \)
    \item \( x^2 \cdot x^7 \)
    \item \( (2a^3)(2a^4) \)
\end{itemize}

\end{document}
